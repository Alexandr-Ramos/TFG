\documentclass[10pt]{book}
\usepackage{eseiaat}


\newcommand{\titulo}{{Real time acoustic analysis \\ and correction }}
\newcommand{\documento}{{Report}}
\newcommand{\autor}{{Alexandr Ramos Sundukov}}
\newcommand{\director}{{Albino Nogueiras Rodríguez}}
\newcommand{\titulacion}{{Bachelor's degree in Audiovisual Systems Engineering}}
\newcommand{\convocatoria}{{Spring, 2025}}


\begin{document}

\frontmatter

\input{portada}

\chapter*{Abstract}


\textit{A short text (250 to 500 words) in which the content and nature of the work is described, focusing particularly on the objectives, methods, results and conclusions of the thesis. The abstract must be in Catalan or Spanish and in English.}
\pagebreak


\tableofcontents
\addcontentsline{toc}{chapter}{List of figures}
\listoffigures
\addcontentsline{toc}{chapter}{List of tables}
\listoftables
\addcontentsline{toc}{chapter}{Abbreviations}
\chapter*{List of abbreviations}

\begin{description}
	\item[ESEIAAT] Escola Superior d’Enginyeries Industrial, Aeroespacial i Audiovisual de Terrassa \cite{eseiaat}.
	\item[RTA] Real Time Analysis
	\item[SMAART] System Measurement Acoustic Analysis Real-time Tool
	\item[FT] Fourier Transform
	\item[FFT] Fast Fourier Transform
	\item[DFT] Discrete Fourier Transform
	\item[RFFT] Real-valued Fast Fourier Transform
	\item[IEC] International Electrotechnical Commision
	\item[RMS] Root Mean Square
	\item[IIR] Infinite Impulse Response
	\item[FIR] Finite Impulse Response
	
\end{description}

\newpage

\mainmatter

\chapter{Introduction}

This chapter presents the objectives and context of the project, which aims to develop a software-based system capable of performing real-time acoustic analysis and applying corrective digital processing to enhance sound quality in specific environments.

\section{Context}

%To understand the proposed solution, we first need to understand what the problem is:

In any space where a sound system is present, there is an environment in which this system operates, as well as an area where the listeners are located. This means that the quality of the sound system and the acoustics of the environment where it is placed work together as a single system that directly affects the sound quality perceived by the listener.

\begin{itemize}
	\item \textbf{About the Sound System:} It is considered that the human ear can perceive sound frequencies between 20 Hz and 20 kHz. 
	%The sound system acts as the source that plays all of these frequencies, and one thing the acoustic environment cannot do is add frequencies that were not already present (assuming the acoustic environment is a fully passive system—if there are other sound sources, they are not considered here). 
	Therefore, a basic requirement for a high-quality sound system is the ability to reproduce the entire audible range with the flattest possible frequency response.%(in other words, equal fidelity across the whole range).
	
	Also, these systems are split across different speakers, which means the listener perceives multiple sound sources. It is considered important to keep all these sources in phase to ensure better sound quality—something that can sometimes be difficult to achieve.
	
	
	\item \textbf{About Acoustic Environment:} The system through which sound travels from the speaker to the listener can be extremely complex. Numerous acoustic phenomena can alter the perceived sound—and sometimes, these phenomena can completely ruin the listening experience. Some of the most important ones include: absorption, reflections, and diffractions, which can lead to reverberation, frequency boosts, cancellations, and resonances. Structural vibrations in the environment can also affect all of these phenomena.
	
	Also, a very important factor is that the environment can be constantly changing, which directly affects these phenomena. For example, air temperature affects the speed of sound, which directly influences these phenomena. Additionally, a window that is open or closed has a very different reflection and absorption coefficient, which directly impacts reflections and reverberation. Similarly, whether a room is empty or filled with people makes a difference—since the human body and clothing absorb sound. For this reason, in professional high-budget theaters and auditoriums, the seating is carefully designed to have a similar absorption coefficient whether or not someone is sitting in the seat. This helps ensure a stable, controlled, and consistent acoustic environment regardless of audience size or configuration. Unfortunately, such solutions are usually very expensive and not always feasible to implement. Even worse, in some cases, the production requirements of certain events may involve removing seats or making significant changes to the environment of a specific venue.
\end{itemize}

Traditionally, achieving a good sound experience required both a high-quality sound system and a well-designed or acoustically treated environment. Nowadays, this is still a best practice to achieve the highest sound quality. However, in recent years, digital technology has also offered new solutions that can further improve the performance of such systems by processing output signals before they reach the speakers—compensating for system deficiencies. These are the kinds of solutions that will be investigated and developed in this project.


\section{Objectives}

The main objective is to develop a custom software-based solution to analyze audio systems and process the signal that will be played through them, in such a way that the processed signal counteracts the system’s issues—improving the listener’s experience in live situations.

This program must be usable by someone who did not participate in its development—not necessarily easy to use, but also not requiring expert-level knowledge.

And also, it is desirable to design the software in a modular and extensible way, allowing future improvements or the integration of new tools without requiring major changes to the existing structure. This will ensure the project remains scalable and adaptable to a wide range of use cases or evolving technical needs.

\section{Report structure}

The report is divided into the following five chapters:

\begin{itemize}
	\item \textbf{State of the art}, where similar solutions will be investigated and their strengths and weaknesses analyzed, in order to guide the development of a useful and effective solution.
	
	\item \textbf{Methodology, consideration and decision on alternative solutions}, where the characteristics of the final solution will be defined based on the previously analyzed alternatives.
	
	\item \textbf{Development of the choosen solution}, where the decisions made during the development process will be explained in detail, along with a full description of the implementation.
	
	\item \textbf{Results}, where the developed solution will be described and tested in a real-world scenario.
	
	\item \textbf{Conclusion}, where the final results will be discussed and a possible future for the project will be outlined.
\end{itemize}

Also, an \textbf{Annex 1: Budget} is included, where the estimated cost of all the components and resources required for the project will be discussed.


\newpage

\chapter{Background and/or status of the matter}

Si s’escau. Situació actual sobre el tema que es pretén estudiar. En aquest apartat, s’hi poden incloure tants subapartats com sigui necessari.

\begin{center}
\vspace{-2mm}
\tikzsetnextfilename{vad_fsa_basico}
\begin{tikzpicture}[node distance=30mm,on grid,auto, scale=1, bend angle=45]
	% \draw[help lines] (0,0) grid (3,2);
	every node/.style={font=\small};
	
	\node (q_init) {start};
	\node (null_init) [right=of q_init] {};
	\node[state, minimum size=15mm] (q_sil) [above right=of null_init] {silence};
	\node[state, minimum size=15mm] (q_voice) [below right=of null_init] {voice};
	\node (null_final) [above right=of q_voice] {};
	\node (q_final) [right=of null_final] {end};
	
	
	\draw[blue, dotted, very thick, ->] (q_init) edge node {} (q_sil);
	\draw[blue, dotted, very thick, ->] (q_init) edge node {} (q_voice);
	\draw[blue, dotted, very thick, ->] (q_sil) edge node {} (q_final);
	\draw[blue, dotted, very thick, ->] (q_voice) edge node {} (q_final);
	
	\path[->,every node/.style={font=\footnotesize}]
	(q_sil) edge [bend left, thick] node {$C_{S,V}$} (q_voice)
	edge [loop above, thick] node {$C_{S}$} ()
	(q_voice) edge [bend left, thick] node  {$C_{V,S}$} (q_sil)
	edge [loop above, thick] node {$C_{V}$} ()
	;
\end{tikzpicture}
\vspace{-2mm}
\end{center}
\newpage

\chapter{Methodology, consideration and decision on alternative solutions}

Passos que cal seguir per aconseguir els objectius del treball, incorporant les tècniques i eines necessàries per resoldre el problema.

Define how program should be.


\newpage

%\input{Files/Plantejament_solucions}
%\newpage

\chapter{Development of the chosen solutions}

El nom d’aquest apartat s'ha de triar en funció del treball que es porti a terme. Aquest apartat pot constar de diversos apartats i subapartats, depenent del criteri de l‘autor o autora i de les consideracions de la tesi que es desenvolupa.

\section{Graphic interface}

About user friendly graphic interface..................................

\section{Settings page}

About parameters that can be set on this page, an why we need them.

\section{Signal Path}
	
How is adquired data from soundcard and how I move this thata inside the program

\section{Acoustics analysis}

The acoustic analysis is devoloped with python librari LIBROSA\cite{librosa}

\subsection{Spectrogram (FT)}

How implemented the spectogram

\subsection{RTA}

Usually, any form of analysis that is performed in real time can be considered \textbf{RTA} (\textit{Real-Time Analysis}). This includes a wide range of operations such as spectrum monitoring, transfer function measurements, phase and coherence analysis, and more—all happening as the signal flows. However, in my experience, in common usage, when someone refers to "RTA", they are often specifically referring to the classic 31-band graphical spectrum display. Also, the data collected on this page is especially important because it will be used to set the correction parameters. For all of that, this page has been named \textbf{RTA}.

Also, there is another conflict. When someone defines the 31 bands and their bandwidth, it is common to use the definition provided in \textbf{IEC 61260}. However, this standard does not mathematically respect the logarithmic spacing between bands. The 31 bands are defined as 1/3 of an octave per band.

\begin{table}[H]
\centering
\caption{Center frequencies for true 1/3 octave bands and IEC 61260 bands} 

	\scalebox{0.53}{
		\begin{tabular}{|c|c|c|c|c|c|c|c|c|c|c|c|}
			\hline
			1/3 octave & 20 & 25.18 & 31.7 & 39.91 & 50.24 & 63.25 & 79.62 & 100.24 & 126.19 & 158.87 & \\ \hline
			IEC 61260 & 20 & 25 & 31.5 & 40 & 50 & 63 & 80 & 100 & 125 & 160 & \\ \hline
			 & & & & & & & & & & & \\ \hline
			1/3 octave & 200 & 251.79 & 316.98 & 399.05 & 502.38 & 632.46 & 796.21 & 1002.37 & 1261.91 & 1588.66 & \\ \hline
			IEC 61260 & 200 & 250 & 315 & 400 & 500 & 630 & 800 & 1000 & 1250 & 1600 & \\ \hline
			 & & & & & & & & & & & \\ \hline
			1/3 octave & 2000 & 2517.85 & 3169.79 & 3990.52 & 5023.77 & 6324.56 & 7962.14 & 10023.74 & 12619.15 & 15886.56 & 20000 \\ \hline
			IEC 61260 & 2000 & 2500 & 3150 & 4000 & 5000 & 6300 & 8000 & 10000 & 12500 & 16000 & 20000 \\ \hline
				
		\end{tabular}
	}
\end{table}

\begin{minted}[label=\texttt{ipython}]{python3}
	"""
	In order to obtain the true 1/3 octave values, the following line of code was used with IPython3.
	The results were rounded to the second decimal place.
	"""
	
	import numpy as np
	print(np.round(np.logspace(np.log10(20), np.log10(20000), 31), 2)) 
	
\end{minted}

On the other hand, this standard is widely used in many professional devices and software. One important example is the DBX 231s graphic equalizer, which is commonly used in analog processing chains.

\begin{figure}[H]
	\centering
	\includegraphics[width=1
	\linewidth]{Figures/DBX_231s.png}
	\caption{Image of the front panel of the DBX 231s \cite{DBX_31s}, where we can observe that the center frequency bands are the same as those defined in IEC 61260.}
	\label{fig:DBX_31s}
\end{figure}

In order to achieve the greatest possible compatibility and coherence with industry standards, I prefer to use the IEC 61260 standard.


The signal path on this page is very similar to the one used on the "FT" page. We have a buffer with two blocks of input data ("In from external device" and "In from system").

First, we copy the buffer data to the "delay buffer", where, if needed, the data will be adjusted to make it coincide with the applied delay. If necessary, the adjustment will use data from previous blocks stored in the same "delay buffer".

I created this buffer with the capacity to store 1 second of data, which can be used to apply a maximum delay of (1 - "Block Size in seconds") seconds.

Once the data in the "delay buffer" is adjusted, we can start applying algorithms to perform the analysis.

The algorithm is based on the calculation of RMS (\textit{root mean square}) to obtain the energy for each frequency band, which is divided using filters for each band.


\begin{figure}[H]
	\begin{center}
		\vspace{-2mm}
		\tikzsetnextfilename{RTA_page_schem}
		\begin{tikzpicture}[node distance=30mm,on grid,auto, scale=1, bend angle=45]
			
			every node/.style={font=\small};
			
			\node (q_init) [draw, rectangle, minimum size=1cm,] {Input Buffer};
			\node (null_init) [right=of q_init] {};
			\node (q_delay) [draw, rectangle, minimum size=1cm, right=of null_init] {Delay Buffer};
			\node (q_ext) [draw, rectangle, minimum size=1cm, above right=of q_delay, xshift=2cm] {Filtering and RMS Calculation of External Input};
			\node (q_in_sys) [draw, rectangle, minimum size=1cm, below right=of q_delay, xshift=2cm] {Filtering and RMS Calculation of Input from System};
			\node (q_diff) [draw, rectangle, minimum size=1cm, right=of q_delay, xshift=4cm] {Difference calculation};	
			
			\draw[blue, very thick, ->] (q_init) edge node {2 channels} (q_delay);
			\draw[blue, dotted, very thick, ->] (q_delay) edge[bend right=10] node {1 channel} (q_ext);
			\draw[blue, dotted, very thick, ->] (q_delay) edge[bend right=10] node {1 channel} (q_in_sys);
			\draw[green, very thick, ->] (q_ext) edge[bend right=10] node {Analysis Data} (q_diff);
			\draw[green, very thick, ->] (q_in_sys) edge[bend right=10] node {Analysis Data} (q_diff);
			
		\end{tikzpicture}
		\vspace{-2mm}
	\end{center}
	\caption{Diagram of the architecture of the RTA page}
\end{figure}


In this case, I'm not using any kind of windowing. As a starting point, I'm using 4th-order IIR (\textit{infinite impulse response}) Butterworth band-pass filters for each band. All these filters are created using the \texttt{scipy.signal} library \cite{scipy_signal}, which returns SOS (\textit{Second-Order Section}) parameters.

\begin{figure}[H]
	\centering
	\caption{Second-Order Sections for IIR filters with their parameters}
	\[
	H(z) = \frac{b_0 + b_1 z^{-1} + b_2 z^{-2}}{1 + a_1 z^{-1} + a_2 z^{-2}}
	\]
\end{figure}

When the program applies each filter to the signal block, it also calculates the RMS and converts it to a logarithmic scale, which will be plotted on the graph and used to calculate the difference graph.

\begin{figure}[H]
	\centering
	\caption{Root Mean Square to calculate energy from filtered signal}
	\[
	RMS = \sqrt{ \frac{1}{N} \sum_{n=0}^{N-1} x^2[n] }
	\]
\end{figure}

At the end, there is: a pause button, which blocks the update function and can be used to pause the graphics; a save button, to store the current values of the difference graph for later use in the correction window; and a time averaging section that works exactly the same as the averaging section from the FT page, except that it does not include frequency averaging (since it doesn't make sense to apply frequency averaging between bands).

\begin{figure}[H]
	\centering
	\includegraphics[width=1
	\linewidth]{Figures/RTA_page.png}
	\caption{Analysis window - RTA page.}
	\label{fig:DBX_31s}
\end{figure}


\subsection{Delay}

Explain delay page

\section{Acoustic correction}

Acoustic correction = DSP, allways have to be somethuing on the output buffer, by default, zeros.

\subsection{Bypass}

How Bypass works, and why it works bad.

\subsection{31 Bars}

Implementation of correction

\section{Integration of monitoring mechanism}

Home page and information that it appears / start, stop streams buttons...

\section{Others}

More problems that I didn't expect, losing time solving them or at least trying to. No more time to implement additional functionalities...
\newpage

\chapter{Results}

About the final program results

There ar not just things to finish or implement, also, it is nedded to solve some actual problems. Most important of them are:

Signal Path glitch and page switching.................................

Less important, are: ......................................................

\begin{itemize}
	\item \textbf{Filter Algorithms:}
	
	\item \textbf{User Limitations:} For example, since we are using the Sounddevice library and managing both inputs as a single input stream, we have the limitation that both channels must come from the same sound card. However, the program currently allows the selection of different sound cards for each input channel.
	
\end{itemize}

\section{Final tests}

For the final test, which involved a real-world scenario, I had the privilege of accessing professional equipment and a real theater. The venue was the \textbf{Teatre del Coro}, located in Sentmenat, Spain, and managed by the non-profit cultural association \textit{Societat Coral Obrera la Gloria Sentmenatenca}. The equipment available for the test was:

\begin{itemize}
	\item \textbf{EVO 4:} External USB audio interface, used as the sound card for the program.
	\item \textbf{Audix TM1:} Measurement microphone, used as the source for the \textbf{Input from System} signal.
	\item \textbf{Mackie SRM-750:} Loudspeaker, responsible for playing the \textbf{Output to System} signal.
	\item \textbf{External Laptop:} Device used to generate the \textbf{External Input} signal.
	\item \textbf{Behringer X32 Compact:} The theater's digital mixing console. It is required to route signals to the speaker. Since it is part of the system, it will also be used to compare the analysis tools of the RTA+C program with the built-in tools of the mixer.
\end{itemize}

The sound card and the measurement microphone were kindly provided by \textit{IMESDE, Integració, Distribució i Enginyeria Escènica, S.L.}, a private company located in La Garriga, Spain. All the devices used can be seen in Figure~\ref{fig:Coro_setup}.

The first step was to place the measurement microphone. It was important to choose a good position where the microphone could capture the sound from a single representative point in the room. The first parameter to define was the microphone height. This venue is very versatile throughout the week, and the seats can be removed. However, events held without the seats typically do not require the use of the theater's sound system. Therefore, it was more appropriate to take measurements with the seats in place. Unfortunately, on the day I was able to perform the measurements, the seats had been partially removed. Nevertheless, I positioned the microphone at the average height of a seated person, as shown in Figure~\ref{fig:Mic_pos1} and Figure~\ref{fig:Mic_pos2}.

\begin{figure}[H]
	\centering
	\includegraphics[width=0.6
	\linewidth]{Figures/Coro_micpos1.jpeg}
	\caption{Microphone height relative to the seat}
	\label{fig:Mic_pos1}
\end{figure}

\begin{figure}[H]
	\centering
	\includegraphics[width=0.6
	\linewidth]{Figures/Coro_micpos2.jpeg}
	\caption{Microphone height = 1.1m}
	\label{fig:Mic_pos2}
\end{figure}

Next, the microphone had to be placed at a representative point in the room. Since I was measuring only one speaker and the program operates in mono, the position had to be one where the speaker had good coverage and where the microphone was as close as possible to the center of the audience area. I decided to place the microphone approximately halfway through the depth of the room. Starting from a position where the speaker was directly in front of the microphone, I shifted it slightly to the right to bring it closer to the actual center of the room. As shown on figure~\ref{fig:Mic_pos3} and figure~\ref{fig:Floor_section}.
\begin{figure}[H]
	\centering
	\includegraphics[width=0.6
	\linewidth]{Figures/Coro_micpos3.jpeg}
	\caption{Microphone position relative to the speaker}
	\label{fig:Mic_pos3}
\end{figure}

\begin{figure}[H]
	\centering
	\includegraphics[width=0.6
	\linewidth]{Figures/Coro_floor_trigo.jpeg}
	\caption{Horizontal distance between speaker and microphone}
	\label{fig:Floor_section}
\end{figure}

Knowing that the speaker is suspended from a rigging bar at a hieght of approximately 6.5 m, and substrating the microphone height, the vertical distance between the two is around 5.4 m. Additionally, the horizontal distance—shown in Figure~\ref{fig:Floor_section}—is 7.03 m. Using basic trigonometric calculations, we can determine that the total distance between the speaker and the microphone is approximately 8.86 m, as illustrated in Figure~\ref{fig:spk-mic}.


\begin{figure}[H]
	\centering
	\includegraphics[width=1
	\linewidth]{Figures/Coro_section_trigo.jpeg}
	\caption{Distance between microphone and speaker}
	\label{fig:spk-mic}
\end{figure}

Once the microphone was positioned, it was time to set up the rest of the equipment, as shown in Figure~\ref{fig:Coro_setup}.

\begin{figure}[H]
	\centering
	\includegraphics[width=1
	\linewidth]{Figures/Coro_setup.jpeg}
	\caption{All equipment set up}
	\label{fig:Coro_setup}
\end{figure}

In order to enable comparison between our software solution and the built-in tools of the digital mixing console (\textit{X32}), all signals must be routed through the \textit{X32}, as it offers advanced routing and distribution capabilities. Additionally, the console must be properly configured to route the signals correctly. Remember that we are using the \textit{EVO 4} as the audio interface for the RTA+C software, as shown in the following connection diagram.

\begin{figure}[H]
	\begin{center}
		\vspace{-2mm}
		\tikzsetnextfilename{connectio_setup}
		\begin{tikzpicture}[node distance=30mm,on grid,auto, scale=1, bend angle=45]
			
			every node/.style={font=\small};
			
			\node (q_ext) [draw, rectangle, minimum size=1cm] {Laptop (source of sound)};
			\node (q_X32_ext) [draw, rectangle, minimum size=1cm, right=of q_ext, xshift=2cm]{X32 - External Input};
			\node (q_evo_ext) [draw, rectangle, minimum size=1cm, right=of q_X32_ext, xshift=2cm]{EVO 4 External Input};
			\node (q_evo_out) [draw, rectangle, minimum size=1cm, below=of q_evo_ext]{EVO 4 Output to System};
			\node (q_X32_out) [draw, rectangle, minimum size=1cm, below=of q_X32_ext]{X32 - Output to System};
			\node (q_spk) [draw, rectangle, minimum size=1cm, below=of q_ext]{Speaker};
			\node (q_mic) [draw, rectangle, minimum size=1cm, below=of q_spk]{Microphone};
			\node (q_X32_in) [draw, rectangle, minimum size=1cm, below=of q_X32_out]{X32 - Input from System};
			\node (q_evo_in) [draw, rectangle, minimum size=1cm, below=of q_evo_out]{EVO 4 Input from System};
			
			\draw[blue, very thick, ->] (q_ext) edge node {} (q_X32_ext);
			\draw[blue, very thick, ->] (q_X32_ext) edge node {} (q_evo_ext);
			\draw[blue, very thick, ->] (q_evo_out) edge node {} (q_X32_out);
			\draw[blue, very thick, ->] (q_X32_out) edge node {} (q_spk);
			\draw[red, very thick, ->] (q_spk) edge [bend left=30] node {System} (q_mic);
			\draw[blue, very thick, ->] (q_mic) edge node {} (q_X32_in);
			\draw[blue, very thick, ->] (q_X32_in) edge node {} (q_evo_in);
			\draw[green, dotted, very thick, ->] (q_X32_ext) edge [bend right=30] node {Bypass or EQ using X32} (q_X32_out);
			\draw[blue,, dotted, very thick, ->] (q_evo_ext) edge [bend right=30] node {Bypass or EQ using RTA+C} (q_evo_out);
			%\draw[blue, very thick, ->] (q_out_bf) edge node {} (q_out_st);
			%\draw[blue, very thick, ->] (q_out_st) edge node {} (q_out);
			
			
		\end{tikzpicture}
		\vspace{-2mm}
	\end{center}
	\caption{System connection overview}
\end{figure}

I compared my system with two basic tools available on the \textit{X32} console:

\begin{itemize}
	\item \textbf{RTA (Real-Time Analysis):} Shown in Figure~\ref{fig:Coro_X32_nontreated}, this tool provides a real-time spectrum display. Although the underlying algorithm is unknown, it is visually the most similar tool to the RTA page in my program.
	\item \textbf{Stereo GEQ (Graphic Equalizer):} A 31-band graphic EQ. Again, the specific algorithm used is not documented, but the concept is equivalent to the DSP window of the program.
\end{itemize}

Once everything was connected, it was time to sit at the control desk (shown in Figure~\ref{fig:Coro_setup}), perform basic checks on the signal path and gain staging, and disable or bypass all unnecessary options and tools on the mixing console. After that, the RTA+C program was launched.

The first step in the program was to configure the sound card and audio parameters using the "\textbf{Settings}" window, as shown in Figure~\ref{fig:Coro_device_settings} and Figure~\ref{fig:Coro_audio_settings}.


\begin{figure}[H]
	\centering
	\includegraphics[width=0.6
	\linewidth]{Figures/Coro_Device_settings.png}
	\caption{Device Settings Window recognizing EVO 4 soundcard}
	\label{fig:Coro_device_settings}
\end{figure}

As we can see in Figure~\ref{fig:Coro_device_settings}, the program automatically recognized the \textit{EVO 4} soundcard without any prior configuration—just by plugging it in and setting the Device Settings window. At this point, I only changed the sample rate to 96 kHz from the Audio Settings page, confirming that the soundcard is compatible with this configuration. The other parameters were left at their default values.

Just to check if the parameters were working properly, I opened the RTA page, but something was not functioning correctly.

\begin{figure}[H]
	\centering
	\includegraphics[width=0.8
	\linewidth]{Figures/Coro_Pink_Bad.png}
	\caption{First RTA measurement over the External Input}
	\label{fig:Coro_Bad_Pink}
\end{figure}

As shown in Figure~\ref{fig:Coro_Bad_Pink}, with pink noise coming from the \textbf{External Input}, the External Input analysis shows a lack of low-frequency energy. I suspected that this might be due to the block size parameter being too small at this sample rate (which means a short analysis time window). To achieve a better representation of low frequencies, I increased the block size parameter, as shown in Figure~\ref{fig:Coro_audio_settings}, to 16,384 samples per block.

\begin{figure}[H]
	\centering
	\includegraphics[width=0.6
	\linewidth]{Figures/Coro_audio_settings.png}
	\caption{Audio Settings Window after changes}
	\label{fig:Coro_audio_settings}
\end{figure}

This block size, at a 96 kHz sampling rate, represents 170.7 ms, which in the worst case corresponds to approximately 3.4 wavelengths at 20 Hz (the lowest frequency considered). With this  setting, we can observe in Figure~\ref{fig:Coro_Good_Pink} that the results are not perfect, but much better. I considered them good enough for this test.

\begin{figure}[H]
	\centering
	\includegraphics[width=0.6
	\linewidth]{Figures/Coro_Pink_Good.png}
	\caption{Second RTA measurement over the External Input}
	\label{fig:Coro_Good_Pink}
\end{figure}

Once this was working properly, I started measuring the delay, which should correspond to the time it takes for sound to travel from the speaker to the microphone. Considering the speed of sound as 340 m/s, this value should represent the distance between the two devices.

\begin{figure}[H]
	\centering
	\includegraphics[width=0.6
	\linewidth]{Figures/Coro_Delay.png}
	\caption{First Delay Measurement}
	\label{fig:Coro_delay1}
\end{figure}

The measurement was not very stable—most iterations gave different values. However, I paused the process and took the most consistent and frequently repeated value. Additionally, I considered the correlation plot to ensure the result was reliable, focusing on cases where the maximum value clearly stood out from the rest. This can be seen in Figure~\ref{fig:Coro_delay1}. 

The measured delay was 36.56 ms, which—considering the speed of sound (340 m/s)—corresponds to a distance of 12.43 m. However, as shown in Figure~\ref{fig:spk-mic}, the actual physical distance between the speaker and the microphone is 8.86 m, which is 3.57 m more than expected.

It is logical to expect a slightly longer delay than the pure acoustic distance, since the signal passes through various systems before reaching the speaker—each of which may introduce some latency. For example, it is known that the mixing console adds approximately 0.3 ms of latency per output channel, which corresponds to only 0.1 m of additional delay. Even if this latency accumulates through multiple stages, it is still not enough to justify the extra 3.57 m (around 10.5 ms).

Because of this significant discrepancy, I continued measuring to investigate further.

\begin{figure}[H]
	\centering
	\includegraphics[width=0.6
	\linewidth]{Figures/Coro_delay_2.png}
	\caption{Second Delay Measurement}
	\label{fig:Coro_delay2}
\end{figure}

After observing the iterations of the plot and the measurement results for a while, I realized that there was a second value that appeared repeatedly across many iterations and also made sense visually on the plot. As shown in Figure~\ref{fig:Coro_delay2}, the plot clearly highlights two peaks with significant correlation values. At the right moment I stopped the measurement, the lower time value among the two had the highest correlation, which corresponded to 32.28 ms. This value translates to a distance of 10.98 m, which is 2.12 m more than the actual physical distance between the speaker and the microphone.

Although not fully conclusive, this second measurement seems more reasonable than the previous one, so I decided to use this value and apply it over the rest of the measurements.

Next step is open the FT page.

\begin{figure}[H]
	\centering
	\includegraphics[width=0.6
	\linewidth]{Figures/Coro_FT_NO_av.png}
	\caption{FT Measurement Without Averaging Applied}
	\label{fig:Coro_FT_no_av}
\end{figure}

Continuing with Pink noise excitation from \textbf{External Input}. I can see the FT page results without applying any avarage option, as shown on Figure~\ref{fig:Coro_FT_no_av}. But the plots are very anarchic, changes a lot between iterations, and is difficult to make a solid interpretaion.

\begin{figure}[H]
	\centering
	\includegraphics[width=0.6
	\linewidth]{Figures/Coro_FT_time_av.png}
	\caption{FT Measurement With Time Averaging Applied}
	\label{fig:Coro_FT_time_av}
\end{figure}

In Figure~\ref{fig:Coro_FT_time_av}, a time average is applied, which greatly improves the consistency between iterations and makes it much easier to interpret the displayed data. However, interpreting the data in the high-frequency range is still somewhat confusing. To address this, I applied frequency averaging.

\begin{figure}[H]
	\centering
	\includegraphics[width=0.6
	\linewidth]{Figures/Coro_FT_WITH_av.png}
	\caption{FT Measurement with Time and Frequency Averaging Applied}
	\label{fig:Coro_FT_av}
\end{figure}

With all averaging options applied, we obtain the result shown in Figure~\ref{fig:Coro_FT_av}. While we lose considerable resolution in the low-frequency range, the high-frequency representation becomes much more visually understandable.

\begin{figure}[H]
	\centering
	\includegraphics[width=0.6
	\linewidth]{Figures/Coro_X32_nontreated.jpeg}
	\caption{X32 RTA Tool with Untreated Input from System Signal}
	\label{fig:Coro_X32_nontreated}
\end{figure}

To check whether the results are reasonably close to reality, we use the \textbf{X32} RTA tool as a reference and compare it with the \textbf{Input from System} analysis. As shown in Figure~\ref{fig:Coro_X32_nontreated}, and comparing it with the FT page of the RTA+C program (see previous figures), the results look very similar—especially the dip between 1000 Hz and 4000 Hz. Therefore, I consider that the FT page works pretty well.

Next step: open the RTA page.

\begin{figure}[H]
	\centering
	\includegraphics[width=0.6
	\linewidth]{Figures/Coro_RTA_Saved.png}
	\caption{Non treated Pink Noise, Saved results..............................................................}
	\label{fig:Coro_RTA_saved}
\end{figure}

\begin{figure}[H]
	\centering
	\includegraphics[width=1
	\linewidth]{Figures/Coro_EQ_from_RTA.png}
	\caption{EQ values applyed form RTA analysis..............................................................}
	\label{fig:Coro_EQ_RTA+C}
\end{figure}

\begin{figure}[H]
	\centering
	\includegraphics[width=0.6
	\linewidth]{Figures/Coro_X32_treatedRTAc.jpeg}
	\caption{Microphone caption with treatment by RTA+C, using X32 analysis tool..................}
	\label{fig:Coro_X32_RTA+C}
\end{figure}

\begin{figure}[H]
	\centering
	\includegraphics[width=0.6
	\linewidth]{Figures/Coro_RTA+EQ_ON.png}
	\caption{Pink Noise with processed signal by RTA+C program..............................................................}
	\label{fig:Coro_RTA_RTA+C}
\end{figure}

\begin{figure}[H]
	\centering
	\includegraphics[width=0.6
	\linewidth]{Figures/Coro_X32_EQ.jpeg}
	\caption{Setting X32 EQ tool with parameters of RTA+C ........................................}
	\label{fig:Coro_X32_EQ}
\end{figure}

\begin{figure}[H]
	\centering
	\includegraphics[width=0.6
	\linewidth]{Figures/Coro_X32_treatedX32.jpeg}
	\caption{Microphone caption with treatment of X32-EQ tool and using X32 analysis tool.........}
	\label{fig:Coro_X32_treatedX32}
\end{figure}

For music, use X32 EQ becouse the glitch ....................................................................

\begin{figure}[H]
	\centering
	\includegraphics[width=0.6
	\linewidth]{Figures/Coro_Music_EQ_X32.png}
	\caption{RTA, using music, and EQ from X32..............................................................}
	\label{fig:Coro_RTA_music}
\end{figure}

\begin{figure}[H]
	\centering
	\includegraphics[width=0.6
	\linewidth]{Figures/Coro_FT_music_EQX32.png}
	\caption{FT, using music, and EQ from X32..............................................................}
	\label{fig:Coro_FT_music}
\end{figure}

Looks preaty nice .......................................................................


\newpage

%\chapter{Annex 1: Budget}

The budget shown in Figure~\ref{fig:Budget} is divided into three sections:


\begin{enumerate}
	\item The first section reflects the cost of time and resources dedicated to the development of the software from February 3rd (the week I began development) to June 8th (the week I stopped development). Additional weeks and hours spent on writing the report are not included.
	
	\item The second section estimates the cost required to complete the development, resulting in a feature-complete, user-friendly, competitive, and marketable program. This is projected as six months of full-time work (not accounting for holidays or vacations). My hourly rate is also increased, since after the approval of this project, I will no longer be considered a student in training, but rather a junior-level graduate professional, able to dedicate myself fully to the task.
	
	\item The third section lists the audio equipment required to deploy the software on-site and operate it effectively. This equipment falls within the professional range, yet remains reasonably priced for its category. All prices were obtained from a trusted reference retailer in the European market—Thomann\cite{thomann}.
\end{enumerate}

It is generally considered that a laptop should be amortized over a period of two years. Assuming that my laptop and its additional peripherals cost approximately 1500€, and should be amortized across 104 weeks, the resulting cost of using the laptop is approximately 14.43€ per week.

Additionally, a General Expenses category has been included, which mainly covers a proportional share of electricity costs and minor office supplies such as pens, paper, pencils, erasers, screen cleaning fluid, and other similar items.


\begin{figure}[H]
	\centering
	\includegraphics[width=0.82
	\linewidth]{Figures/Budget.png}
	\caption{Budget for the entire project}
	\label{fig:Budget}
\end{figure}


%\newpage

%\chapter{Anàlisi i valoració de les implicacions ambientals i socials}

Also think that I don't need.

Anàlisi i valoració resumides de com el projecte i/o l’estudi dut a terme en el treball té en compte i/o millora diferents aspectes ambientals i de temàtica social.

%\newpage

\chapter{Conclusions}

After more than four months of work on this project, not all objectives were achieved. The overall goal of developing a software-based solution for acoustic analysis and correction was not fully completed. Several tools are still pending implementation (such as phase analysis, RT60 measurement, waterfall diagram, dynamic EQ to constantly adapt to changes in the environment, among others), and some of the tools already implemented could benefit from improvements (such as the \textbf{GUI} or the filters used). In addition, there are two major issues that must be resolved for the software to be fully usable: GUI freezing and signal desynchronization in the \textbf{DSP} window.

Even so, if we ignore the two main issues, the rest of the developed program can be considered a success. As shown in the final test section, it is functional and performs quite well. Moreover, a solid foundational architecture was developed, which makes it easy to add or modify tools and functionalities in the future. This extensible and modular design is, in my opinion, one of the most solid and valuable achievements of the project.

On the other hand, if I—or someone else—were to continue this project with another four months of development time, I would recommend starting the program from scratch. The same structure and toolset could be implemented following the same logic, but with a serious rethinking of the GUI system (to avoid unexplained bugs and improve code organization) and the signal path management (to eliminate the glitch effect). If these two key issues were properly addressed in a new version of the program, it would result in a very strong foundation to keep building on—allowing for easier expansion, refinement of existing tools, and unlocking the full potential of the project.

From a market perspective, this project—if fully developed with all its planned functionalities working correctly—could be a strong candidate for commercial release. As discussed in the \textbf{State of the art} chapter, there are already several existing solutions that offer many of the tools proposed in this project. However, these alternatives are often either too expensive (because they are hardware-based), or they lack flexibility, such as not integrating analysis and correction in the same environment, or offering fewer tools overall.

If additional features were implemented beyond the current scope—such as multichannel system analysis and correction (for stereo setups or distributed measurement points) or a microphone calibration system—this solution could not only remain at a much more competitive price point compared to hardware-based options, but also become one of the most complete and practical software tools available on the market.

Ultimately, beyond the technical aspects, this project has represented a valuable learning journey. It has provided deep insight into real-time audio processing, signal analysis, and software design challenges—especially considering that I had never developed a graphical user interface before. Facing this for the first time added an extra layer of complexity to the project, but also made the experience more enriching. The process of solving unexpected issues, iterating on solutions, and building a complex tool from scratch has contributed significantly to my personal and professional growth. Regardless of its current limitations, the project lays down a strong conceptual and technical foundation that can serve as the starting point for more advanced developments in the future—and that, in itself, is a success.

\vfill
\section*{Acknowledgements}

\textit{I would like to thank my project director for the guidance and support provided throughout the development of this work.}

\textit{I am also grateful to the Societat Coral Obrera la Glòria Sentmenatenca for allowing me to use their theater for the final testing, and to IMESDE, Integració, Distribució i Enginyeria Escènica, S.L. for lending me part of their equipment and offering flexibility in my schedule, which allowed me to dedicate more time to this project.}

\textit{Finally, I want to thank my parents and friends for their constant support throughout the entire process.}



\newpage

\newpage
\printbibliography[heading=bibintoc]

\backmatter
\pagenumbering{Roman}
\chapter{Annex 1: Budget}


\chapter{Annex 2: Complete Code}

\begin{minted}[label=\texttt{divisiones.py}]{python3}
	"""
	Biblioteca con definiciones importantes para la división de números.
	Se incluyen las funciones divideSiDivisible() y cocienteModulo().
	"""
	
	def divideSiDivisible(nume, deno):
	"""
	Si nume es divisible por deno, devuelve la división
	entera. Si no lo es, devuelve None.
	"""
	
	if not nume % deno:
	return nume // deno
	
	
	def cocienteModulo(nume, deno):
	"""
	Devuelve el conciente entero y el resto de
	la divisi
	ón entera (mod) de dos números.
	"""
	
	return nume // deno, nume % deno
	
\end{minted}


\inputminted[label={src/divisions.py},firstline=6,lastline=13]{python}{src/divisions.py}


\end{document}
