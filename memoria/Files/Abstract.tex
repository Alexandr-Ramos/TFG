\chapter*{Abstract}
\addcontentsline{toc}{chapter}{Abstract}


%\textit{Text breu (entre 250 a 500 paraules) en què s'informa del contingut i la naturalesa del treball, on s'hi fan constar especialment els objectius, els mètodes, els resultats i les conclusions del treball. El resum ha de ser en català o castellà i anglès.}

%Actualment hi han 272 paraules de (250 o 500)

\textit{This project addresses a common challenge faced by audio engineers and technicians: adjusting sound systems based on the acoustic properties of a given environment. While there are existing commercial solutions that offer advanced analysis tools, this work proposes the development of a custom, software-based alternative using open-source technologies.}
	
\textit{The proposed solution is implemented in Python and designed to perform real-time acoustic analysis and correction. It integrates several scientific libraries, including NumPy and SciPy for signal processing, and uses Sounddevice for real-time audio input and output. The program interacts with the environment through a basic hardware setup consisting of a microphone, a speaker, and a sound card.}
	
\textit{A graphical user interface is developed using Tkinter and Matplotlib to ensure usability and ease of interaction. This graphical user interface allows users to configure audio parameters, select input channels, visualize data, and monitor system behavior in real time. Features include Fourier transformation, filter-based processing, and measurement tools such as delay, phase, and frequency response analysis.}
	
\textit{Throughout the development, special attention is given to flexibility and accessibility. The software is intended to run on standard hardware under Linux (specifically Ubuntu 22.04.5 LTS), making it usable in both professional and home studio environments.}

\textit{Finally, the project reflects on the difficulties encountered during development, acknowledging that not all objectives have been fully achieved. It discusses the main technical and practical challenges, and offers constructive criticism to guide future iterations. Despite these limitations, the result is a solid foundation on which to build. While the tool is not yet fully functional, it provides a working base from which the missing features can be implemented and the existing ones improved.}