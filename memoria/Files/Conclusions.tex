\chapter{Conclusions}

The project has resulted in a functional and extensible software-based platform for acoustic analysis and correction. Although not all planned objectives were fully achieved, the main goals were addressed, and the current version provides a solid foundation with several useful tools already integrated. These include real-time analysis, a spectrograph, delay measurement, and a DSP module with a 31-band equalizer capable of processing an external audio signal. Moreover, the system is adaptable to a wide range of audio equipment — from low-budget setups to high-end professional systems.

One of the intended applications of this solution is not only to improve the acoustic quality of a given environment, but also to adapt to changes in that environment over time. Situations such as opening windows, removing seats, or variations in audience presence can significantly alter the acoustic response of a space. The modular design and potential for dynamic correction features—like adaptive equalization—make this tool especially suitable for dealing with such evolving conditions, even in modest setups.

Some advanced features, such as phase analysis, RT60 measurement, and waterfall diagrams, remain as potential future extensions. Additionally, some existing tools could benefit from refinement, including improvements to the graphical user interface and filter design. These enhancements would help increase usability and performance, but do not compromise the core functionality already in place.

The development process also highlighted a couple of important technical challenges—specifically, occasional GUI freezing and audio desynchronization in the DSP module. These are known limitations and will require targeted debugging and restructuring. However, they do not affect all use cases and can be addressed in future iterations thanks to the clear modular structure of the program.

From a market perspective, this project—if fully developed with all its planned functionalities working correctly—could be a strong candidate for commercial release. As discussed in the \textbf{State of the art} chapter, there are already several existing solutions that offer many of the tools proposed in this project. However, these alternatives are often either too expensive (because they are hardware-based), or they lack flexibility, such as not integrating analysis and correction in the same environment, or offering fewer tools overall.

If additional features were implemented beyond the current scope—such as multichannel system analysis and correction (for stereo setups or distributed measurement points) or a microphone calibration system—this solution could not only remain at a much more competitive price point compared to hardware-based options, but also become one of the most complete and practical software tools available on the market.

Ultimately, beyond the technical aspects, this project has represented a valuable learning journey. It has provided deep insight into real-time audio processing, signal analysis, and software design challenges—especially considering that I had never developed a graphical user interface before. Facing this for the first time added an extra layer of complexity to the project, but also made the experience more enriching. The process of solving unexpected issues, iterating on solutions, and building a complex tool from scratch has contributed significantly to my personal and professional growth. The project lays down a strong conceptual and technical foundation that can serve as the starting point for more advanced developments in the future—and that, in itself, is a success.

%After more than four months of work on this project, not all objectives were achieved. The overall goal of developing a software-based solution for acoustic analysis and correction was not fully completed. Several tools are still pending implementation (such as phase analysis, RT60 measurement, waterfall diagram, dynamic EQ to constantly adapt to changes in the environment, among others), and some of the tools already implemented could benefit from improvements (such as the \textbf{GUI} or the filters used). In addition, there are two major issues that must be resolved for the software to be fully usable: GUI freezing and signal desynchronization in the \textbf{DSP} window.

%Even so, if we ignore the two main issues, the rest of the developed program can be considered a success. As shown in the final test section, it is functional and performs quite well. Moreover, a solid foundational architecture was developed, which makes it easy to add or modify tools and functionalities in the future. This extensible and modular design is, in my opinion, one of the most solid and valuable achievements of the project.

%On the other hand, if I—or someone else—were to continue this project with another four months of development time, I would recommend starting the program from scratch. The same structure and toolset could be implemented following the same logic, but with a serious rethinking of the GUI system (to avoid unexplained bugs and improve code organization) and the signal path management (to eliminate the glitch effect). If these two key issues were properly addressed in a new version of the program, it would result in a very strong foundation to keep building on—allowing for easier expansion, refinement of existing tools, and unlocking the full potential of the project.

%From a market perspective, this project—if fully developed with all its planned functionalities working correctly—could be a strong candidate for commercial release. As discussed in the \textbf{State of the art} chapter, there are already several existing solutions that offer many of the tools proposed in this project. However, these alternatives are often either too expensive (because they are hardware-based), or they lack flexibility, such as not integrating analysis and correction in the same environment, or offering fewer tools overall.

%If additional features were implemented beyond the current scope—such as multichannel system analysis and correction (for stereo setups or distributed measurement points) or a microphone calibration system—this solution could not only remain at a much more competitive price point compared to hardware-based options, but also become one of the most complete and practical software tools available on the market.

%Ultimately, beyond the technical aspects, this project has represented a valuable learning journey. It has provided deep insight into real-time audio processing, signal analysis, and software design challenges—especially considering that I had never developed a graphical user interface before. Facing this for the first time added an extra layer of complexity to the project, but also made the experience more enriching. The process of solving unexpected issues, iterating on solutions, and building a complex tool from scratch has contributed significantly to my personal and professional growth. Regardless of its current limitations, the project lays down a strong conceptual and technical foundation that can serve as the starting point for more advanced developments in the future—and that, in itself, is a success.

\vfill
\section*{Acknowledgements}

\textit{I would like to thank my project director for the guidance and support provided throughout the development of this work.}

\textit{I am also grateful to the Societat Coral Obrera la Glòria Sentmenatenca for allowing me to use their theater for the final testing, and to IMESDE, Integració, Distribució i Enginyeria Escènica, S.L. for lending me part of their equipment and offering flexibility in my schedule, which allowed me to dedicate more time to this project.}

\textit{Finally, I want to thank my parents and friends for their constant support throughout the entire process.}


