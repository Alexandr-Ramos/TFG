\chapter{Introduction}

This chapter presents the objectives and context of the project, which aims to develop a software-based system capable of performing real-time acoustic analysis and applying corrective digital processing to enhance sound quality in specific environments.

\section{Context}

%To understand the proposed solution, we first need to understand what the problem is:

In any space where a sound system is present, there is an environment in which this system operates, as well as an area where the listeners are located. This means that the quality of the sound system and the acoustics of the environment where it is placed work together as a single system that directly affects the sound quality perceived by the listener.

\begin{itemize}
	\item \textbf{About the Sound System:} It is considered that the human ear can perceive sound frequencies between 20 Hz and 20 kHz. 
	%The sound system acts as the source that plays all of these frequencies, and one thing the acoustic environment cannot do is add frequencies that were not already present (assuming the acoustic environment is a fully passive system—if there are other sound sources, they are not considered here). 
	Therefore, a basic requirement for a high-quality sound system is the ability to reproduce the entire audible range with the flattest possible frequency response.%(in other words, equal fidelity across the whole range).
	
	Also, these systems are split across different speakers, which means the listener perceives multiple sound sources. It is considered important to keep all these sources in phase to ensure better sound quality—something that can sometimes be difficult to achieve.
	
	
	\item \textbf{About Acoustic Environment:} The system through which sound travels from the speaker to the listener can be extremely complex. Numerous acoustic phenomena can alter the perceived sound—and sometimes, these phenomena can completely ruin the listening experience. Some of the most important ones include: absorption, reflections, and diffractions, which can lead to reverberation, frequency boosts, cancellations, and resonances. Structural vibrations in the environment can also affect all of these phenomena.
	
	Also, a very important factor is that the environment can be constantly changing, which directly affects these phenomena. For example, air temperature affects the speed of sound, which directly influences these phenomena. Additionally, a window that is open or closed has a very different reflection and absorption coefficient, which directly impacts reflections and reverberation. Similarly, whether a room is empty or filled with people makes a difference—since the human body and clothing absorb sound. For this reason, in professional high-budget theaters and auditoriums, the seating is carefully designed to have a similar absorption coefficient whether or not someone is sitting in the seat. This helps ensure a stable, controlled, and consistent acoustic environment regardless of audience size or configuration. Unfortunately, such solutions are usually very expensive and not always feasible to implement. Even worse, in some cases, the production requirements of certain events may involve removing seats or making significant changes to the environment of a specific venue.
\end{itemize}

Traditionally, achieving a good sound experience required both a high-quality sound system and a well-designed or acoustically treated environment. Nowadays, this is still a best practice to achieve the highest sound quality. However, in recent years, digital technology has also offered new solutions that can further improve the performance of such systems by processing output signals before they reach the speakers—compensating for system deficiencies. These are the kinds of solutions that will be investigated and developed in this project.


\section{Objectives}

The main objective is to develop a custom software-based solution to analyze audio systems and process the signal that will be played through them, in such a way that the processed signal counteracts the system’s issues—improving the listener’s experience in live situations.

This program must be usable by someone who did not participate in its development—not necessarily easy to use, but also not requiring expert-level knowledge.

And also, it is desirable to design the software in a modular and extensible way, allowing future improvements or the integration of new tools without requiring major changes to the existing structure. This will ensure the project remains scalable and adaptable to a wide range of use cases or evolving technical needs.

\section{Report structure}

The report is divided into the following five chapters:

\begin{itemize}
	\item \textbf{State of the art}, where similar solutions will be investigated and their strengths and weaknesses analyzed, in order to guide the development of a useful and effective solution.
	
	\item \textbf{Methodology, consideration and decision on alternative solutions}, where the characteristics of the final solution will be defined based on the previously analyzed alternatives.
	
	\item \textbf{Development of the choosen solution}, where the decisions made during the development process will be explained in detail, along with a full description of the implementation.
	
	\item \textbf{Results}, where the developed solution will be described and tested in a real-world scenario.
	
	\item \textbf{Conclusion}, where the final results will be discussed and a possible future for the project will be outlined.
\end{itemize}

Also, an \textbf{Annex 1: Budget} is included, where the estimated cost of all the components and resources required for the project will be discussed.

